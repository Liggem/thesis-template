\documentclass[
	a4paper,
	11pt,
	% oneside,
	twoside,
	% openright,
	final,
	% draft,
]{includes/mpreport}

\usepackage[type=bachelor, lang=de]{includes/wdok-title}

\usepackage[utf8]{inputenc}
\usepackage{textcomp}
\usepackage{mathptmx}
\usepackage{bibentry}
\usepackage{helvet}
\usepackage{courier}
\usepackage[hyperfootnotes=false, plainpages=false, pdfpagelabels]{hyperref}	
\usepackage{makeidx}
\usepackage{listings}	
\usepackage{color}
\usepackage{natbib}
\usepackage{fancyhdr}
\usepackage[smaller]{acronym}
\usepackage{tabularx}
\usepackage{ntheorem}
\usepackage{array}
\usepackage[justification=raggedleft]{caption}
\usepackage{relsize}
\usepackage{graphicx}
\usepackage{stmaryrd}
\usepackage{subfig}
\usepackage{longtable}
\usepackage{tabularx}
\usepackage{verbatim}
\usepackage{csvsimple}
\usepackage{longtable}
\usepackage[breakall]{truncate}
\usepackage[ngerman]{babel}
\usepackage{acronym}
\usepackage{pdfpages}
\usepackage[normalem]{ulem}
\makeatletter
\renewcommand{\lst@MakeCaption}[1]{
  \lst@ifdisplaystyle
  \ifx #1t%
  \ifx\lst@@caption\@empty\expandafter\lst@HRefStepCounter \else
  \expandafter\refstepcounter
  \fi {lstlisting}%
  \ifx\lst@label\@empty\else \label{\lst@label}\fi
  \let\lst@arg\lst@intname \lst@ReplaceIn\lst@arg\lst@filenamerpl
  \global\let\lst@name\lst@arg \global\let\lstname\lst@name
  \lst@ifnolol\else
  \ifx\lst@@caption\@empty
  \ifx\lst@caption\@empty
  \ifx\lst@intname\@empty \else \def\lst@temp{ }%
  \ifx\lst@intname\lst@temp \else
  \addcontentsline{lol}{lstlisting}\lst@name
  \fi\fi
  \fi
  \else
  \addcontentsline{lol}{lstlisting}%
  {\protect\numberline{\thelstlisting}\lst@@caption}%
  \fi
  \fi
  \fi
  \ifx\lst@caption\@empty\else
  \lst@IfSubstring #1\lst@captionpos
  {\begingroup \let\@@vskip\vskip
    \def\vskip{\afterassignment\lst@vskip \@tempskipa}%
    \def\lst@vskip{\nobreak\@@vskip\@tempskipa\nobreak}%
    \par\@parboxrestore\normalsize\normalfont % \noindent (AS)
    \ifx #1t\allowbreak \fi
    \ifx\lst@title\@empty
    \settoheight{\mp@capht}%
    {\parbox[b]{.9\mp@margwd}{\raggedleft\sffamily\slshape\small%
        \lst@makecaption\fnum@lstlisting\lst@caption}}%

    \hspace{-\mp@margwd}%
    \vspace{-\mp@capht}%
    \parbox[t]{.9\mp@margwd}{\raggedleft\sffamily\slshape\small%
      \lst@makecaption\fnum@lstlisting\lst@caption}%

    % \lst@makecaption\fnum@lstlisting\lst@caption % (AS)
    \else
    \lst@maketitle\lst@title % (AS)
    \fi
    \ifx #1b\allowbreak \fi
    \endgroup}{}%
  \fi
  \fi
}
\makeatother


\setcounter{tocdepth}{2}
\setcounter{secnumdepth}{4}

% Einrückung der ersten Zeile eines Absatzes
% \parindent3mm

\newcommand{\thesisTitle}{Software-definierte dynamische Thesis}
\newcommand{\thesisAuthor}{John Doe}
\newcommand{\thesisSupervisor}{Prof. Dr. Bernd Freisleben}
\newcommand{\thesisReader}{Mitarbeiter Name}
\newcommand{\thesisDate}{01. Januar 1970}

% Title page information
\title{\thesisTitle{}}
\author{\thesisAuthor{}}
\supervisor{\thesisSupervisor{}}
\reader{\thesisReader{}}

% Setzen der Meta-Daten
\hypersetup{
	pdftitle={\thesisTitle{}},
	pdfauthor={\thesisAuthor{}},
	pdfkeywords={Thesis, Abschlussarbeit, \thesisTitle{}, \thesisAuthor{}, \thesisSupervisor{}},
	pdfsubject={\thesisTitle{}}
}

% Konfiguration der Listings im Dokument
% more info: http://www.theofel.de/archives/2004/10/source_code_in.html

\definecolor{darkgreen}{cmyk}{1,0,1,0.15} % Dark green is more readable.
\renewcommand{\ttdefault}{pcr}\par
\lstset{
	numbers=left, 				% show line numbers on the left
	stepnumber=1, 				% show every  line number
	numberstyle=\small, 			% font sitze of line numbers
	numbersep=5pt, 				% separation to the text
	language=Bash,				% language of source
	basicstyle=\footnotesize\ttfamily,	% generic style
	stringstyle=\color{blue},		% style of strings
	emphstyle=\color{red},			% sytle of ???
	commentstyle=\color{darkgreen},		% style of comments
	keywordstyle=\bfseries,			% style of keywords
	tabsize=4,				% indentation
	showspaces=false,			% show spaces
	showtabs=false,				% show tabs
	showstringspaces=false,			% show spaces in strings
	breaklines=true,			% breaks long lines
	frame=tbl 				% can be: none, single, trbl (or trBL, ...), shadowbox
}

% Festlegung Art der Zitierung - Havardmethode: Abkuerzung Autor + Jahr
\bibliographystyle{alpha}

% Zeilenabstand
\usepackage{setspace}
%\doublespacing    % doppelzeilig oder
\onehalfspacing  % anderthalbzeilig

% Hurenkinder und Schusterjungen werden vermieden
\clubpenalty = 10000
\widowpenalty = 10000
\displaywidowpenalty = 10000

% Hier beginnt das eigentliche Dokument
\begin{document}
\maketitle

\newpage
\thispagestyle{empty}
\mbox{}
\newpage

% Ab jetzt römische Seitenzahlen
\renewcommand{\thepage}{\Roman{page}}
\setcounter{page}{1}


\phantomsection
\addcontentsline{toc}{chapter}{Abstract}
%!TEX root = ../thesis.tex
\newpage

\chapter*{Abstract}\label{intro}


\clearpage

\phantomsection
\addcontentsline{toc}{chapter}{Inhaltsverzeichnis}
\tableofcontents

\cleardoublepage
% Ab jetzt arabische Seitenzahlen
\renewcommand{\thepage}{\arabic{page}}
\setcounter{page}{1}

%!TEX root = ../thesis.tex

\newpage

\chapter{Einleitung} % (fold)
\label{cha:einleitung}

Hier beginnt die Arbeit und Paper werden zitiert \cite{baumgaertner2015misuse}.
% \include{chapters/02_Grundlagen}
% \include{chapters/03_RelatedWork}
% \include{chapters/04_Design}
% \include{chapters/05_Implementierung}
% \include{chapters/06_Evaluation}
% \include{chapters/07_Zusammenfassung}


% 2 Seiten frei, damit der umbruch stimmt
\newpage
\thispagestyle{empty}
\mbox{}
\newpage
\thispagestyle{empty}
\mbox{}

% Im Anhang werden wieder Römische Seitenzahlen verwendet
\renewcommand{\thepage}{\Roman{page}}
\setcounter{page}{1}
% \appendix

\phantomsection
\addcontentsline{toc}{chapter}{Literaturverzeichnis}
\bibliography{literature}
\newpage


\phantomsection
\addcontentsline{toc}{chapter}{Tabellenverzeichnis}
\listoftables
\newpage


\phantomsection
\addcontentsline{toc}{chapter}{Abbildungsverzeichnis}
\listoffigures
\newpage


\phantomsection
\addcontentsline{toc}{chapter}{Listings}
\lstlistoflistings
\newpage


\phantomsection
\addcontentsline{toc}{chapter}{Appendix}
\chapter*{Appendix}


\phantomsection
\addcontentsline{toc}{chapter}{Abkürzungsverzeichnis}
\chapter*{Abkürzungsverzeichnis} % (fold)
\label{sec:abkurzungen}

\begin{acronym}
  \acro{TLS}{Transport Layer Security}
\end{acronym}

% section abkurzungen (end)



%!TEX root = ../thesis.tex

\phantomsection
\addcontentsline{toc}{chapter}{Eidesstattliche Erklärung}
\chapter*{Eidesstattliche Erklärung}

Ich erkläre, dass ich meine Thesis \emph{\thesisTitle{}} selbstständig und ohne Benutzung anderer als der angegebenen Hilfsmittel angefertigt habe, und dass ich alle Stellen, die ich wörtlich oder sinngemäß
aus Veröffentlichungen entnommen habe, als solche kenntlich gemacht habe. Die Arbeit hat bisher in gleicher oder ähnlicher Form oder auszugsweise noch keiner Prüfungsbehörde vorgelegen.\\
\\
Ich versichere, dass die eingereichte schriftliche Fassung der auf dem beigefügten Medium gespeicherten Fassung entspricht.\\
\\
\\
\\
\begin{table}[h]
\begin{tabularx}{\columnwidth}{XlX}
Marburg, den \thesisDate{} &  &              \\ \cline{3-3} 
                           &  & \multicolumn{1}{r}{\small{{\thesisAuthor{}}}}
\end{tabularx}
\end{table}

\end{document}
